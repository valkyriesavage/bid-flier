\documentclass[foldmark,10pt,a4paper,notumble]{leaflet}

\usepackage{graphicx}
\renewcommand{\sfdefault}{phv}
\renewcommand{\familydefault}{\sfdefault}
\newcommand{\name}[1]{#1,}
\newcommand{\email}[1]{\emph{#1}\\}
\newcommand{\website}[1]{\emph{#1}\\[3pt]}

\usepackage{titlesec}
\titleformat*{\subsection}{\normalsize\bfseries}

\begin{document}
\begin{center}
\includegraphics{bid.png}\\
\Large{Berkeley Institute of Design}\\
\Large{http://bid.berkeley.edu}\\
\end{center}
The Berkeley Institute of Design (BiD) is a research group that fosters a new and deeply interdisciplinary approach to design for the 21st century.  We are an active research group with a focus that spans human-computer interaction (HCI), mechanical design, education, architecture, and art practice.\\

\subsection{For More Information}
John Canny\\
Computer Science\\
Director, Berkeley Institute of Design\\
\emph{jfc@cs.berkeley.edu}

\subsection{Affiliate Faculty}
Alice Agogino, Mechanical Engineering\\
\emph{agogino@berkeley.edu}\\
\\
Eric Paulos, Computer Science\\
\emph{paulos@berkeley.edu}\\
\\
Bj\"orn Hartmann, Computer Science\\
\emph{bjoern@eecs.berkeley.edu}\\
\\
Greg Niemeyer, Art Practice\\
\emph{niemeyer@berkeley.edu}\\
\\
Kimiko Ryokai, School of Information\\
\emph{kimiko@ischool.berkeley.edu}\\
\newpage
\fontsize{3mm}{3.5mm}\selectfont

\subsection{Designing Video-Based Interactive Instructions}
\name{Peggy Chi}
\email{peggychi@cs.berkeley.edu}
\website{http://www.cs.berkeley.edu/$\sim$peggychi/}
When attempting to accomplish unfamiliar tasks, people often search for online tutorials to follow instructions. We design video-based recording, editing, and playback tools optimized for creating and consuming interactive tutorials from author demonstrations. Our approaches acquire videos and high-level events that are important to a learner and automatically generate novel instructional formats. A series of systems to support this vision includes: MixT [UIST'12], DemoCut [UIST'13], Kinectograph [CHI'13], DemoWiz[CHI'14], and an ongoing project for authoring motion illustrations.

\subsection{Resistance as a Resource for Design}
\name{Laura Devendorf}
\email{ldevendorf@ischool.berkeley.edu}
\website{http://artfordorks.com}
My research blends computer science, art history, and science and technology studies to explore how design can provoke collaborative relationships between people, technology, and non-human things. Collaborative relationships require care and compromise and I explore them through the development of interfaces that resist a person�s familiar practices. My studies of resistance in the design of mobile apps, wearable technology, and digital fabrication show how collaborating with resistant technology can provoke curiosity, inspiration, and enchanting encounters with everyday things.

\subsection{Environmental Impacts of 3D Printing}
\name{Jeremy Faludi}
\email{faludi@berkeley.edu}
\website{http://www.faludidesign.com}
3D printing is revolutionizing manufacturing; will it also revolutionize the ecological impacts of making things? Or will it create more problems than it solves? Some colleagues and I in the UC Berkeley mechanical engineering department performed a full scope 3 cradle to gate life-cycle assessment of two 3D printers and a �traditional� CNC milling machine, to compare their eco-impacts. This resulted in a paper in Rapid Prototyping Journal, due for publication in 2014.

\subsection{Human-Inspired Technology}
\name{Shiry Ginosar}
\email{shiry@eecs.berkeley.edu}
\website{http://www.eecs.berkeley.edu/$\sim$shiry}
Humans are highly evolved beings with computational capabilities that modern algorithms can rarely achieve. This is particularly true in the area of complex learning and visual perception. I am interested in modeling human performance in different areas and designing technology inspired by these abilities.

\subsection{Synthesizing explanations of online example code}
\name{Andrew Head}
\email{andrewhead@eecs.berkeley.edu}
\website{http://andrewhead.info}
Programmers use the web to learn and clarify details as they write code. But online information can be fragmented, unreliable, incomplete, or written for readers with a different background. I develop systems that synthesize explanations for under-explained code on the fly.  I explore what makes effective components of explanations, and techniques for building explanation generators.

\subsection{Interactive machine learning}
\name{Biye Jiang}
\email{bjiang@cs.berkeley.edu}
\website{http://byeah.github.io/}
I am interested in building toolkits for modern data scientists who will usually work on prototyping new models or running experiments on large scale dataset. Our methodology includes but not limit to using hardware accelerations like GPU, providing implementation framework for machine learning algorithms, building visual interface for real-time control and monitoring. Boosting low-level machine performance and improving human productivity are both important for modern data analytic tasks. Therefore my research is trying to bridge the gap between users and the complex machine learning systems.

\subsection{Design Roadmapping}
\name{Euiyoung Kim}
\email{euiyoungkim@berkeley.edu}
\website{http://best.berkeley.edu/best-research/design-roadmap/}
While product and technology roadmaps have been well-formalized in terms of their structures, methodologies, and frameworks, design roadmaps have not been explicitly explored nor studied from either an academic or industry practice standpoint. Strategies that revolve around the holistic experience of products are more likely to be successful in today�s market. As first step, we develop a design driven roadmapping process as a new way of preparing future product concepts and define elements and steps.

\subsection{Design thinking for social impact}
\name{Julia Kramer}
\email{j.kramer@berkeley.edu}
I conduct research at the intersection of design thinking and global development. I investigate the design methods and tools that designers use when working on social impact projects, and I seek to understand how these methods are related to the context in which the designers are working. I also explore design theory and methodology more generally through my work on theDesignExchange (www.thedesignexchange.org), an online repository and ontology of design methods useful for designers across a variety of disciplines.

\subsection{Integrating Existing Objects with Digital Designs}
\name{Jingyi Li}
\email{noon@berkeley.edu}
\website{http://jingyi.me}
Design tools for digital fabrication have opened a wide array of opportunities for creating new objects with functional and aesthetic properties. Currently, we are exploring leveraging digital fabrication to modify or augment existing objects. In Makers� Marks, physically annotated sculptures serve as tangible blueprints for new functional objects, and in Banksybot, found objects are canvases for surface engravings and decorations.


\subsection{ESP (Example-based Sensor Predictions)}
\name{David A. Mellis}
\email{mellis@berkeley.edu}
\website{http://web.media.mit.edu/~mellis/}
This project helps novices make sophisticated use of sensors in interactive projects through the application of machine learning. With our system, experts author example code for particular applications. Our system generates a GUI which allows end-users to supply their own training data, calibrate their sensors, and tune the machine learning algorithms for incorporation in their own projects. This example-centric approach allows to provide a customized, usable interface across a wide range of real-time sensing applications.
% (w/ Ben Zhang, Audrey Leung, and Bjoern Hartmann)

\subsection{Internet-based Collaboration}
\name{Mitar Milutinovi\'c}
\email{mitar@tnode.net}
\website{http://mitar.tnode.com}
My research interests are e-democracy, deliberative democracy,
collective intelligence, trust networks, group decision support systems,
collaboration tools, peer-to-peer and distributed systems, wireless
mesh/community networks, organic/self-healing technologies. Currently, I
am working on a novel platform for enriching the experience of open
access scholarly literature called PeerLibrary (http://peerlibrary.org/)
and on a voting scheme which uses a social network between voters to
find a better overall group decision.

\subsection{Actuated Tensegrity Structures for Dynamic Locomotion}
\name{Andrew Sabelhaus}
\email{apsabelhaus@berkeley.edu}
\website{http://tinyurl.com/TensegrityRobots}
Combining the popular mechatronics paradigms of soft robotics and bio-inspiration, actuated Tensegrity structures offer opportunities for unique movement and locomotion methods. Tensegrity robots are comprised of purely tensile and compressive elements (cables and rods) that change shape when their tensions are adjusted. Based on prior work that used machine learning techniques to develop control laws for such actuation, this research seeks to create a physical realization of a Tensegrity robot.

\subsection{Fabbed To Sense: Co-Design of Geometry and Sensing Algorithms for Interactive Objects}
\name{Valkyrie Savage}
\email{valkyrie@eecs.berkeley.edu}
\website{http://valkyriesavage.com}
Digital fabrication tools like 3D printers and laser cutters are often used in the process of prototyping interactive objects. However, they are often used to make static parts, like cases or mounts. We have developed a framework linking pre-fabrication digital geometry and knowledge of fabricated material properties to desired sensing techniques; thus we can leverage 3D printed plastics or vinyl-cut foils directly as senseable elements. We call this ``Fabbing to Sense'', and over the course of my research we have developed three instances of this technique: Midas [UIST12], Lamello [CHI15], and Sauron [UIST13].

\subsection{Identify misunderstanding in students and delivering guidance to fix them}
\name{Kristin Stephens-Martinez}
\email{ksteph@cs.berkeley.edu}
\website{http://www.cs.berkeley.edu/$\sim$ksteph/}
We propose a model that uses student incorrect constructed responses to identify conceptual misunderstandings. We do this by labeling the popular wrong answers with the misunderstandings they indicate, then use co-occurrence to predict other answers' labels. This approach leverages a small amount of human effort to seed an automated procedure. Our approach involves much less effort than inspecting all answers and substantially outperforms an uninformed baseline. Our next steps involve using the misunderstanding model to deliver guidance messages to students in hopes to fix their misunderstandings.

\vspace{12pt}
Latest update: \today

\end{document}
