\documentclass[foldmark,10pt,a4paper,notumble]{leaflet}

\usepackage{graphicx}
\renewcommand{\familydefault}{\sfdefault}

\usepackage{titlesec}
\titleformat*{\subsection}{\normalsize\bfseries}

\begin{document}
\begin{center}
\includegraphics{bid.png}\\
\Large{Berkeley Institute of Design}\\
\Large{http://bid.berkeley.edu}\\
\end{center}
The Berkeley Institute of Design (BiD) is a research group that fosters a new and deeply interdisciplinary approach to design for the 21st century.  We are an active research group with a focus that spans human-computer interaction (HCI), mechanical design, education, architecture, and art practice.\\

\subsection{For More Information}
John Canny\\
Computer Science\\
Director, Berkeley Institute of Design\\
\emph{jfc@cs.berkeley.edu}

\subsection{Affiliate Faculty}
Alice Agogino, Mechanical Engineering\\
\emph{agogino@berkeley.edu}\\
\\
Maneesh Agrawala, Computer Science\\
\emph{maneesh@cs.berkeley.edu}\\
\\
Bj\"orn Hartmann, Computer Science\\
\emph{bjoern@eecs.berkeley.edu}\\
\\
Greg Niemeyer, Art Practice\\
\emph{niemeyer@berkeley.edu}\\
\\
Kimiko Ryokai, School of Information\\
\emph{kimiko@ischool.berkeley.edu}\\
\newpage
\fontsize{3mm}{3.5mm}\selectfont

\subsection{DERRICK}
\emph{Derrick Cheng, derrick@eecs.berkeley.edu}\\
\emph{http://linkedin.com/chengderrick}\\
I work with Professor Canny in behavioral data mining. We are currently working on a project in Urban Analytics, in which we are analyzing community feedback on topics related to urban planning and development. 
In addition to this work, I am also involved in a project called Kinectograph - an automatic body tracking camera for DIY tutorials.

\subsection{Fostering Deep Engagements with Technology}
\emph{Laura Devendorf, ldevendorf@ischool.berkeley.edu}\\
\emph{http://artfordorks.com}\\
Working at the intersection of the art and computer science, I design and study technologies for creativity, self expression, reflection, and discovery.

\subsection{Turning Sustainability From a Burden into an Innovation Tool}
\emph{Jeremy Faludi, faludi@berkeley.edu}\\
\emph{http://www.faludidesign.com}\\
Sustainability has historically been seen by designers and engineers as a burden--a further constraint on them.  This new design method instead uses sustainability points of view as tools for innovation.  It uses a combination of whole-systems and life-cycle thinking to break down assumptions and lead engineers to reformulate problems for radical reductions in environmental impacts while also improving functionality, business viability, and user experience.

\subsection{Human-Centric User Research to Identify Disruptive Opportunties in Convergent Paper and Digital Use}
\emph{Euiyoung Kim, euiyoungkim@berkeley.edu}\\
\emph{http://euiyoungkim.wordpress.com}\\
Although digital devices have their own unique features that differentiate them from other
tangible types of resources for reading, writing and sketching, a majority of people still prefers traditional
paper media as it provides better user experiences in many aspects: readability, portability, ease of
making annotations, shared reading, tactile sensory experiences, etc. This research will identify
barriers and opportunities for paper-like features based on various human-centered design methods and
explore a new product concept driven by this research. Based on this design research, we propose to
design conceptual prototypes along with use scenarios. 

\subsection{Using AI to Extract Information from Charts}
\emph{Nicholas Kong, nkong@cs.berkeley.edu}\\
\emph{http://eecs.berkeley.edu/$\sim$nkong}\\
Visualizations and charts must be carefully designed for their audience, but even the best designed charts are not ideal for every viewer. We apply image and natural language processing techniques to charts to extract information that can help improve the utility of visualizations for users. We have used these techniques for chart redesign and overlays for charts, and are working on associating text with marks (such as bars or lines) in the chart.

\subsection{Internet-based Collaboration}
\emph{Mitar Milutinovi\'c, mitar@tnode.net}\\
\emph{http://mitar.tnode.com}\\
My research interests are e-democracy, deliberative democracy,
collective intelligence, trust networks, group decision support systems,
collaboration tools, peer-to-peer and distributed systems, wireless
mesh/community networks, organic/self-healing technologies. Currently, I
am working on a novel platform for enriching the experience of open
access scholarly literature called PeerLibrary (http://peerlibrary.org/)
and on a voting scheme which uses a social network between voters to
find a better overall group decision.

\subsection{PABLO}
\emph{Pablo Paredes, paredes@eecs.berkeley.edu}\\
\emph{http://eecs.berkeley.edu/$\sim$paredes}\\
My research focuses on Human Potential Realization. HCI has been supporting efforts to help people realize goals that are externally imposed and time bounded through CSCW and Mass Media technology. Recently Persuasive and Behavior Change Technologies have been supporting short term intrinsic goals. However there seems to be a gap in technology development for long-term intrinsic goals, those so-called life goals, such as having a healthy, stress-free life, being a good father, or fulfilling a career, etc. I am researching the type of sensing, feedback, intervention and identity technologies, needed to help people realize such life goals. I am currently exploring machinima as a way to deliver interactive movies, sensor-less sensing, as a way to non-invasively detect affect and live journal data mining for auto biographical narrative study to support identity change.

\subsection{Improving Editing Interfaces for Audio Stories}
\emph{Steve Rubin, srubin@cs.berkeley.edu}\\
\emph{http://ssrubin.com}\\
Producers of radio shows, podcasts, and audiobooks face
several challenges when editing raw footage into a final story. One
challenge is that of adding musical scores to stories. Our UnderScore
system analyzes music and allows producers to automatically add music
to emphasize important parts of the their stories. With UnderScore and
other tools, we aim to allow producers to think of and edit their
stories in terms of high-level content as opposed to low-level
waveforms.

\subsection{Actuated Tensegrity Structures for Dynamic Locomotion}
\emph{Andrew Sabelhaus, apsabelhaus@berkeley.edu}\\
\emph{http://tinyurl.com/TensegrityRobots}\\
Combining the popular mechatronics paradigms of soft robotics and bio-inspiration, actuated Tensegrity structures offer opportunities for unique movement and locomotion methods. Tensegrity robots are comprised of purely tensile and compressive elements (cables and rods) that change shape when their tensions are adjusted. Based on prior work that used machine learning techniques to develop control laws for such actuation, this research seeks to create a physical realization of a Tensegrity robot.

\subsection{Sauron: Embedded Single-Camera Sensing of Printed Physical User Interfaces}
\emph{Valkyrie Savage, valkyrie@eecs.berkeley.edu}\\
\emph{http://valkyriesavage.com}\\
3D printers enable designers to rapidly produce models of future products.  Today these physical prototypes are mostly \emph{passive}.  Our goal is to enable designers to turn models produced on commodity 3D printers into interactive objects with minimal assembly or instrumentation.  We present Sauron, an embedded machine vision-based system for sensing human input on physical controls like buttons, sliders, and joysticks.  Sauron automatically modifies the interior geometry of a designer's object so that, when printed, all components can be sensed by adding a single inward-pointing camera.

\subsection{MOOC Analytics}
\emph{Kristin Stephens-Martinez, ksteph@cs.berkeley.edu}\\
\emph{http://www.cs.berkeley.edu/$\sim$ksteph/}\\
Massive Open Online Courses (MOOCs) have tens of thousands of students enrolled in an online course at once. With so many more students than physical classrooms and lots of new kinds of information recorded, how can we use it to make things better? My goal is to find and automate ways to find interesting things in MOOC data. Then I will also find how best to visualize that information for teachers.

\subsection{ANUJ}
\emph{Anuj Tewari, anuj@eecs.berkeley.edu}\\
\emph{http://cs.berkeley.edu/$\sim$anuj}\\
Lack of proper English instructions is a major problem for a variety of populations around the world. This lack of exposure poses various serious problems, including a barrier to entry into mainstream society. At the Berkeley Institute of Design we explore a variety of topics like mobile phone games for language learning in developing countries, pronunciation feedback for Hispanic children and question-answering technologies for preschoolers. Details on some of these projects can be found here: http://www.cs.berkeley.edu/$\sim$anuj/language\_learning\_games

\subsection{Ingenuity Lab: Making and Engineering through Design Challenges at a Science Center}
\emph{Jennifer Wang, jennifer\_wang@berkeley.edu}\\
\emph{http://best.berkeley.edu/$\sim$jen}\\
My research studies how to increase access and interest in engineering through tinkering spaces at public science centers, in particular at the Ingenuity Lab at the Lawrence Hall of Science. I implement an innovative cross-community design of Ingenuity Lab programs involving engineers, students, and educators and analyze its impact on the learner experience. Through observations, surveys, and interviews, I study learners' engineering-as-"tinkering" experience, persistence in the activities, and potential consequences for long-term interest.
\end{document}
